% \def\year{2020}\relax
% File: formatting-instruction.tex
\documentclass[letterpaper]{article} % DO NOT CHANGE THIS
\usepackage{aaai20}  % DO NOT CHANGE THIS
\usepackage{times}  % DO NOT CHANGE THIS
\usepackage{helvet} % DO NOT CHANGE THIS
\usepackage{courier}  % DO NOT CHANGE THIS
\usepackage[hyphens]{url}  % DO NOT CHANGE THIS
\usepackage{graphicx} % DO NOT CHANGE THIS
\urlstyle{rm} % DO NOT CHANGE THIS
\def\UrlFont{\rm}  % DO NOT CHANGE THIS
\usepackage{graphicx}  % DO NOT CHANGE THIS
\frenchspacing  % DO NOT CHANGE THIS
\setlength{\pdfpagewidth}{8.5in}  % DO NOT CHANGE THIS
\setlength{\pdfpageheight}{11in}  % DO NOT CHANGE THIS
% \nocopyright
% PDF Info Is REQUIRED.
% For /Author, add all authors within the parentheses, separated by commas. No accents or commands.
% For /Title, add Title in Mixed Case. No accents or commands. Retain the parentheses.
\pdfinfo{/Title gm_hmm
} %Leave this	
\newcommand{\citet}[1]{\citeauthor{#1} \shortcite{#1}}
\newcommand{\citep}{\cite}
\newcommand{\citealp}[1]{\citeauthor{#1} \citeyear{#1}}
%%%%%%%%%%%%%%%%%%%%%%%% 

\usepackage{tikz}
\usetikzlibrary{calc,shapes,positioning}
\usetikzlibrary{arrows}
\newcommand{\midarrow}{\tikz \draw[-triangle 90] (0,0) -- +(.1,0);}
% Be sure to use PDxF Latex
\pdfoutput=1

\usepackage[latin1]{inputenc}




\usepackage{caption}
\usepackage{bm}
\newcommand{\ubar}[1]{\mkern2mu\underline{\mkern-2mu #1\mkern-2mu}\mkern2mu}
% \allowdisplaybreaks
\usepackage{mystyle}
\newcommand{\ubm}[1]{\ubar{\bm{#1}}}
\newcommand{\ubmr}[2]{\ubar{\bm{#1}}^{#2}}


% \newcommand{\bmtr}[3]{\bm{#1}^{(#3)}_{#2}}

\newcommand{\bmtr}[3]{\bm{#1}^{#3}_{#2}}
\newcommand{\smtr}[3]{{#1}^{#3}_{#2}}


\usepackage{amsmath,graphicx}
% format A4
% \usepackage{vmargin}
% \setpapersize{A4} 

\RequirePackage{algorithm}
\RequirePackage{algorithmic}

% Attempt to make hyperref and algorithmic work together better:
% \newcommand{\theHalgorithm}{\arabic{algorithm}}


\graphicspath{{./images/}}


% \hypersetup{  
% bookmarks=true,
% backref=true,
% pagebackref=false,
% colorlinks=true,
% linkcolor=blue,
% citecolor=red,
% urlcolor=blue,
% pdftitle={Generative Model HMM},
% pdfauthor={Dong Liu},
% pdfsubject={}
% }


\setcounter{secnumdepth}{2} %May be changed to 1 or 2 if section numbers are desired.

% The file aaai20.sty is the style file for AAAI Press 
% proceedings, working notes, and technical reports.
% 
\setlength\titlebox{2.5in} % If your paper contains an overfull \vbox too high warning at the beginning of the document, use this
% command to correct it. You may not alter the value below 2.5 in

\title{Powering Hidden Markov Model by Generative Models}
\author{
  Dong Liu, Antoine Honor�, Saikat Chatterjee, and Lars~K. Rasmussen\\
  KTH Royal Institute of Technology, Stockholm, Sweden \\
  E-mail: \{doli, honore, sach, lkra\}@kth.se}


% \name{
% Dong Liu,
% Minh Th�nh Vu,
% Saikat Chatterjee,
% and Lars~K. Rasmussen
% }

%   \address{
%   KTH Royal Institute of Technology, Stockholm, Sweden \\
%   E-mail: \{doli, mtvu, sach, lkra\}@kth.se}

\begin{document}

\maketitle
\begin{abstract}
  to be written...
\end{abstract}

\section{introduction}
Sequential data modeling is a challenging topic in pattern recognition and machine learning, since for many applications, the assumption of independent and identically distributed data points is too strong to model data properly. Hidden Markov model (HMM) is classic way to model sequential data. HMM has been widely used in different practical problems, and especially is known for applications in reinforcement learning [], natural language modeling [, biological sequence analysis such as proteins and DNA [], etc.

HMM provides an framework for sequential data modeling. But for ..., a probabilistic model is used to describe how each state of HMM should represent a part of sequential data. The typical way is to use a Gaussian mixture model (GMM) per state of HMM, where GMMs are used to represent the relationship between states of HMM and sequential data input. A large amount of research on GMM based HMM (GMMHMM) has been carried out[] in literature.

Given the success of GMMHMM, it is not efficient for modeling data in nonlinear manifold. Research attempts of training HMM with neural networks have been done to boost the modeling capacity of HMM. A successful work is to bring restrictive Boltzmann machine (RBM) \cite{Hinton2012} into HMM learning \cite{hinton2012deepSpeech}. In RBM based HMM, the hierarchical learning scheme is used, where a stack of RBMs are trained in unsupervised fashion , and then the stack of RBMs with a final softmax layer are used to mapping sequential input signal to states of a HMM. ... talk about some other DNNHMMs.

Introduce our model...

Contribution ...


\section{Generator-mixed HMM}

\begin{figure}[!h]
  \centering
  \begin{tikzpicture}
    \tikzstyle{enode} = [thick, draw=black, ellipse, inner sep = 1pt,  align=center]
    \tikzstyle{nnode} = [thick, rectangle, rounded corners = 2pt,minimum size = 0.8cm,draw,inner sep = 2pt]
    \node[enode] (g1) at (-0.5,1.8) {$p(\bm{x}| s=1; \bm{\Phi}_{1})$};
    \node[enode] (g2) at (-0.5,0.5) {$p(\bm{x}| s=2; \bm{\Phi}_{2})$};
    \node[enode] (gs) at (-0.5, -1.8) {$p(\bm{x}| s=|\Ss|; \bm{\Phi}_{|\Ss|})$};
    \node[enode] (x) at (4.5,1.5){$\ubm{x}\sim p(\ubm{x};\bm{H})$};

    \draw[dotted,line width=2pt] (0,-0.3) -- (0,-1.2);
    \filldraw[->] (1.9, 0.5)circle (2pt) --  (x) ;
    \draw[->] (g1) -- (1.8, 1.8);
    \draw[->] (g2) -- (1.8, 0.5);
    \draw[->] (gs) -- (1.8, -1.8);

    \begin{scope}[xshift=0.5cm, thick, every node/.style={sloped,allow upside down}]
      \node[nnode] (m) at (3.5,-2) {Memory};
      \node[nnode] (a) at (3.5,-0.5) {$\bm{A}$};

      \draw (2.1,0.9)-- (2.2, 0.);
      \draw (2.2,0.)-- node {\midarrow} (2.2,-2);
      \draw (2.2,-2)-- (m);
      \draw (m)-- (5, -2);
      \draw (5, -2)-- node {\midarrow} (5 ,-0.5);
      \draw (5, -0.5) -- (a);
      \draw (a)-- node {\midarrow} (2.2, -0.5);
      \node at (4.8, -1) {$s_{t}$};
      \node at (2.56, -0.25) {$s_{t+1}$};
    \end{scope}
  \end{tikzpicture}
  \caption{HMM model illustration.}\label{fig:hmm}
  \vspace{0.1cm}
\end{figure}

Our framework is an Hidden Markov Model (HMM). An HMM model $\bm{H}$ defined in a hypothesis space $\Hh$, i.e. $\bm{H} \in \Hh$, is capable to model time-span signal $\ubar{\bm{x}} = \left[ \bm{x}_1, \cdots, \bm{x}_T\right]^{\intercal}$, where $\bm{x}_t\in \RR^{N}$ is the signal at time $t$, $[\cdot]^{\intercal}$ means transpose, and $T$ denote the time length. Using HMM for signal representation or generating is illustrated in \ref{fig:hmm}. The assumption is that different instant signal of $\ubar{\bm{x}}$ is generated by different source, while the source selection is done by a discrete hidden markov process. We define the hypothesis set of HMM as $\Hh := \{\bm{H} | \{\Ss, \bm{q}, \bm{A}, p(\bm{x}|{s}; \bm{\Phi}_{s})\}$, where:
\begin{itemize}
\item $\Ss$ is the set of states of HMM $\bm{H}$;
\item $\bm{q} = \left[ q_1, q_2, \cdots, q_{|\Ss|}\right]^\intercal$ initial distribution of HMM $\bm{H}$ with $|\Ss|$ is cardinality of $\Ss$, $q_k = p(s=k)$ for random state variable $s$. In the following of this paper, we use $s_t$ to denote the state $s$ at time $t$.
\item $\bm{A}$ is the transition matrix for the HMM $\bm{H}$ of size $|\Ss| \times |\Ss|$, $\bm{A}_{i,j} = p(s_{t+1}=j|s_{t}=i)$.
\item For given hidden state $s$, the observable signal density is $p({\bm{x}}|{s};\bm{\Phi}_{s})$, where $\bm{\Phi}_{s}$ is the parameter set that defines this conditional probabilistic model.
\end{itemize}

In the framework of HMM, at each time instance, signal $\bm{x}_t$ is generated by density $p(\bm{x}_t| s_t; \bm{\Phi}_{s_t})$, and $s_t$ is decided by the hidden markov process. This process gives us the probabilistic model $p(\ubm{x};\bm{H})$
\subsection{Emission Model of GenHMM}

\begin{figure}[!th]
  \centering
  \begin{tikzpicture}
    \tikzstyle{enode} = [thick, draw=black, ellipse, inner sep = 2pt,  align=center]
    \tikzstyle{nnode} = [thick, rectangle, rounded corners = 2pt,minimum size = 0.8cm,draw,inner sep = 2pt]
    \node[enode] (z1) at (-1.2,1.8) {$\bm{z}\sim p_{s,1}(\bm{z})$};
    \node[nnode] (g1) at (1,1.8) {$\bm{g}_{s,1}$};
    \node[enode] (z2) at (-1.2,0.5){$\bm{z}\sim p_{s,2}(\bm{z})$};
    \node[nnode] (g2) at (1,0.5) {$\bm{g}_{s,2}$};
    \node[enode] (zK) at (-1.2,-1.8) {$\bm{z}\sim p_{s,K}(\bm{z})$};
    \node[nnode] (gs) at (1, -1.8) {$\bm{g}_{s,K}$};
    \node[enode] (x) at (4.5,0){$\bm{x}\sim p(\bm{x}| s; \bm{\Phi}_{s})$};

    \draw[dotted,line width=2pt] (0,-0.3) -- (0,-1.2);
    \filldraw[->] (1.9, 0.5)circle (2pt) --  node[above=0.2]{${\kappa}\sim \bm{\pi}_{s}$} (x)  ;
    \draw[->] (z1) -- (g1);
    \draw[->] (g1) -- (1.8, 1.8);

    \draw[->] (z2) -- (g2);
    \draw[->] (g2) -- (1.8, 0.5);

    \draw[->] (zK) -- (gs);
    \draw[->] (gs) -- (1.8, -1.8);
  \end{tikzpicture}
  \caption{Source of state $s$ in GenHMM.}
  \label{fig:gen-mix}
  \vspace{0.1cm}
\end{figure}

In this section, we introduce a neural network based generator-mixed HMM model. We term it as GenHMM. GenHMM has a hidden markov process as typical HMM, while the probabilistic model of GenHMM for each hidden state is induced by mixture of neural network based generators. For given state $s\in \Ss$, we define its probabilistic model in GenHMM as
\begin{equation}
  p(\bm{x}| s; \bm{\Phi}_{s}) = \sum_{\kappa=1}^{K}\pi_{s, \kappa} p(\bm{x}| s, \kappa; \bm{\theta}_{s, \kappa}),
\end{equation}
where $\kappa$ is a random variable following a categorical distribution, and $\pi_{s, \kappa} = p(\kappa | s; \bm{\Phi}_{s, \kappa})$. Naturally $\sum_{\kappa = 1}^{K} \pi_{s, \kappa}= 1$. Denote $\bm{\pi}_{s} = [\pi_{s,1}, \pi_{s,2}, \cdots, \pi_{s,K}]$,
$\bm{\theta}_s = \left\{ \bm{\theta}_{s, \kappa}| \kappa = 1, 2, \cdots, K \right\}$. 
We define $p(\bm{x}| s, \kappa; \bm{\theta}_{s, \kappa})$ as induced distribution by a generator $\bm{g}_{s,\kappa}: \RR^{N}\rightarrow\RR^{N}$, such that $\bm{x}=\bm{g}_{s, \kappa}(\bm{z})$, where $\bm{z}\sim p_{s,\kappa}(\bm{z})$. By change of variable, we have
\begin{equation}\label{eq:changel-variable}
  p(\bm{x}| s, \kappa; \bm{\theta}_{s, \kappa}) = p_{s,\kappa}(\bm{z})\bigg| \det\left( \pd{\bm{g}_{s,\kappa}(\bm{z})}{\bm{z}} \right)\bigg|^{-1}.
\end{equation}

For a state $s$ of GenHMM, the induced probability distribution is illustrated in Figure~\ref{fig:gen-mix}.

\subsection{Learning in EM framework}

For a given dataset, we denote its empirical distribution by $\hat{p}(\ubm{x}) = \frac{1}{R}\sum_{r=1}^{R} \delta_{\ubmr{x}{r}}(\ubm{x})$, where superscipt $(\cdot)^{r}$ denotes the index of sequential signal. The natural question is how to use GenHMM to represent the dataset. Alternative, we are looking for the answer to question:
\begin{equation}
  \umin{\bm{H}\in \Hh} KL(\hat{p}(\ubm{x})\| p(\ubm{x};\bm{H}))
\end{equation}
where $KL(\cdot\|\cdot)$ denotes the Kullback-Leibler divergence. The KL divergence problem can be boiled down to a maximizing a loglikelihood problem:
\begin{equation}\label{eq:ml-of-hmm}
  \uargmax{\bm{H} \in \Hh} \sum_{r=1}^{R}\log\,p(\ubmr{x}{r}; \bm{H}).
\end{equation}

Since model $\bm{H}$ contains hidden sequential variable $\ubm{s}$ and $\ubm{\kappa}$, we can not directly solve the maximum likelihood problem in \ref{eq:ml-of-hmm}. We use expectation maximization (EM) to address the hidden variable problem by
\begin{itemize}
\item E-step: % the posterior probability of $\ubm{s}$:
  % \begin{equation}
  %   p(\ubm{s}|\ubm{x})
  % \end{equation}
  The ``expected likelihood'' function:
  \begin{equation}\label{eq:em-q-funciton}
    \Qq(\bm{H}; \bm{H}^{\mathrm{old}}) = \EE_{\hat{p}(\ubm{x}),p(\ubm{s},\ubm{\kappa}| \ubm{x}; \bm{H}^{\mathrm{old}})}\left[ \log\,p(\ubm{x}, \ubm{s}, \ubm{\kappa}; \bm{H})\right],
  \end{equation}
  where $\EE_{\hat{p}(\ubm{x}),p(\ubm{s},\ubm{\kappa}| \ubm{x}; \bm{H}^{\mathrm{old}})}\left[ \cdot\right]$ denotes the expectation operator by distribution $\hat{p}(\ubm{x})$ and $p(\ubm{s},\ubm{\kappa}| \ubm{x}; \bm{H}^{\mathrm{old}})$.
\item M-step: the optimization step:
  \begin{equation}\label{eq:em-m-opt}
    \umax{\bm{H}} \Qq(\bm{H}; \bm{H}^{\mathrm{old}})
  \end{equation}
\end{itemize}


The \eqref{eq:em-m-opt} can be reformulated as:
\begin{align}\label{eq:m-step-subs}
  &\umax{\bm{H}} \Qq(\bm{H}; \bm{H}^{\mathrm{old}}) \nonumber \\
  =&\umax{\bm{q}}\Qq(\bm{q}; \bm{H}^{\mathrm{old}}) + \umax{{A}}\Qq({A}; \bm{H}^{\mathrm{old}}) 
     + \umax{\bm{\Phi}}\Qq(\bm{\Phi}; \bm{H}^{\mathrm{old}})
\end{align}
where
\begin{align}
  \Qq(\bm{q}; \bm{H}^{\mathrm{old}}) 
  &=\EE_{\hat{p}(\ubm{x}),p(\ubm{s},\ubm{\kappa}| \ubm{x}; \bm{H}^{\mathrm{old}})} \left[ \log\,p({s}_{1})  \right] \nonumber\\
  &= \EE_{\hat{p}(\ubm{x}),p(\ubm{s}| \ubm{x}; \bm{H}^{\mathrm{old}})} \left[ \log\,p({s}_{1})  \right]\label{eq:init-distribution-update}\\
  \Qq(\bm{A}; \bm{H}^{\mathrm{old}}) &=\EE_{\hat{p}(\ubm{x}),p(\ubm{s}| \ubm{x}; \bm{H}^{\mathrm{old}})}\left[ \sum_{t}\log\,p({s}_{t+1}|{s}_{t}; \bm{H}) \right] \label{eq:transition-update}\\
  \Qq(\bm{\Phi}; \bm{H}^{\mathrm{old}}) &= \EE_{\hat{p}(\ubm{x}),p(\ubm{s},\ubm{\kappa}| \ubm{x}; \bm{H}^{\mathrm{old}})} \left[ \log\,p(\ubm{x}, \ubm{\kappa}| \ubm{s}; \bm{H}) \right]\label{eq:generative-model-update}
\end{align}

We can see that the solution of $H$ depends on the posterior probability $p(\ubm{s}| \ubm{x}; \bm{H})$. Though the evaluation of posterior according to Bayesian theorem is simple, the computation complexity of $p(\ubm{s}| \ubm{x}; \bm{H})$ grows exponentially with the length of $\ubm{s}$. Therefore, we would employ Forward/Backward algorithm \cite{} to do the posterior computation efficiently. The marginal $p(s_t| \ubm{x}; \bm{H})$ is also efficiently computed as the joint posterior. 


With such a model GenHMM at hand, the remaining problems are:
\begin{itemize}
\item How to realize GenHMM by neural networks to have high modeling capacity?
\item How to train GenHMM to sovle problem in \eqref{eq:ml-of-hmm} using practical algorithm?
\item Would the training of GenHMM converges?
\end{itemize}
We would answer these questions in the following section.



\section{Practical Solution for GenHMM}

In this section, we provide answers to the questions at the end of last section.



\subsection{Modeling $\bm{g}_{s,\kappa}$ by a flow model }
We use feed-forward neural network to implement every generator $\bm{g}_{s,\kappa}$. For notation simplicity, denote $\bm{g}_{s,\kappa}=\bm{g}_{s,\kappa}$. Define  a feed-forward neural network $\bm{g}_{s,\kappa}$ that has multiple hidden layers
$\bm{g}_{s,\kappa}=\bm{g}_{s,\kappa}^{[L]}\circ \bm{g}_{s,\kappa}^{[L-1]}\circ \cdots
\circ \bm{g}_{s,\kappa}^{[1]}$ and is invertible $\bm{f}_{s,\kappa}=\bm{g}_{s,\kappa}^{-1}$, where $\circ$ denotes mapping concatenation. Then the signal flow can be depicted as
\begin{equation*}
  \vspace{-8pt}
  \centering
  \begin{tikzpicture}
    \node (z) at (0,0) {};
    \node at ($(z)-(0.5,0)$){$\bm{z}=\bm{h}_0$};
    \node (xi1) at (1.5,0) {$\bm{h}_1$};
    \node (xi2) at (3,0) {};
    \node (xi3) at (4.5,0){};
    \node (x) at (6,0) {};
    \node at ($(x)+(0.5,0)$){${\bm{x}} = \bm{h}_L$};
    \draw[->] ($(z) + (0.3,0.1)$) -- node[above]{$\bm{g}_{s,\kappa}^{[1]}$} ($(xi1)+(-0.3,0.1)$); 
    \draw[->] ($(xi1)-(0.3,0.1)$) -- node[below]{${\bm{f}}_{s,\kappa}^{[1]}$}($(z) - (-0.3,0.1)$);
    \draw[->] ($(xi1) + (0.3,0.1)$) -- node[above]{$\bm{g}_{s,\kappa}^{[2]}$} ($(xi2)+(-0.3,0.1)$); 
    \draw[->] ($(xi2)-(0.3,0.1)$) -- node[below]{${\bm{f}}_{s,\kappa}^{[2]}$}($(xi1) - (-0.3,0.1)$);
    \draw[->] ($(xi3) + (0.3,0.1)$) -- node[above]{$\bm{g}_{s,\kappa}^{[L]}$} ($(x)+(-0.3,0.1)$); 
    \draw[->] ($(x)-(0.3,0.1)$) -- node[below]{${\bm{f}}_{s,\kappa}^{[L]}$}($(xi3) - (-0.3,0.1)$);
    \draw[dotted,line width = 0.3 mm] (xi2) -- (xi3);
  \end{tikzpicture},
\end{equation*}
where $\bm{g}_{s,\kappa}^{[l]}$ and $\bm{f}_{s,\kappa}^{[l]}$ are the $l$-th layer of $\bm{g}_{s,\kappa}$ and $\bm{f}_{s,\kappa}$, respectively. In a feed-forward neural network, if every layer is invertible,
the full feed-forward neural network is invertible. The inverse
function is given by $\bm{z}=\bm{f}_{s,\kappa}(\bm{x})$. Flow-based
network, proposed in \cite{DBLP:journals/corr/DinhKB14}, is such a
feed-forward neural network, which is further improved in subsequent works \cite{2016arXiv160508803D, 2018arXiv180703039K}. It also has additional advantages as efficient Jacobian computation and low computational complexity. %Computation of inverse mapping for flow-based network and determinant of Jacobin were addressed in the literature. 

For a flow-based neural network architecture, let us assume that
the feature $\bm{h}_l$ at the $l$'th layer has two subparts as
$\bm{h}_l = [\bm{h}_{l,a}^{T} \, , \, \bm{h}_{l,b}^{T}]^{T}$ where
$(\cdot)^{T}$ denotes transpose operation. Then considering $\bm{h}_0 = \bm{z}$, we have the following forward and inverse relations between $(l-1)$'th and $l$'th layers:
\begin{equation}\label{eq-gl}
  \begin{array}{l}
    \bm{h}_{l-1} =
    \begin{bmatrix}
      \bm{h}_{l-1,a}\\
      \bm{h}_{l-1,b}
    \end{bmatrix}
    =
    \begin{bmatrix}
      \bm{h}_{l,a}\\
      \bm{m}_a(\bm{h}_{l,a})\odot \bm{h}_{l,b} + \bm{m}_b(\bm{h}_{l,a})
    \end{bmatrix},\vspace{10pt}\\
    \bm{h}_{l} =
    \begin{bmatrix}
      \bm{h}_{l,a}\\
      \bm{h}_{l,b}
    \end{bmatrix}
    =
    \begin{bmatrix}
      \bm{h}_{l-1,a}\\
      \left(  \bm{h}_{l-1,b} - \bm{m}_b(\bm{h}_{l-1,a}) \right)\oslash \bm{m}_a(\bm{h}_{l-1,a}) 
    \end{bmatrix}, \\
  \end{array}  
\end{equation}
where $\odot$ denotes element-wise product, $\oslash$ denotes
element-wise division, and $\bm{m}_a(\cdot), \bm{m}_b(\cdot)$ can be
complex non-linear mappings (implemented by neural networks).
For the flow-based neural network, the determinant of Jacobian matrix is
\begin{equation}\label{eq:cat-jacobian}
  \begin{array}{rl}
    \mathrm{det}(\nabla{\bm{f}_{s,\kappa}}) & = \prod_{l=1}^L \det (\nabla{\bm{f}_{s,\kappa}^{[l]}}),
  \end{array}
\end{equation}
where $\nabla{f_{s,\kappa}^{[l]}}$ is the Jacobian of the transformation from the $l$-th layer to the $(l-1)$-th layer, i.e., the inverse transformation. We compute the determinate of the Jacobian matrix as
\begin{align}\label{eq-hl-determinate}
  \det (\nabla{f_{s,\kappa}^{[l]}})& = \det \left[  \pd{\bm{h}_{l-1}}{\bm{h}_l} \right] \nonumber\\
                                   & = \det
                                     \begin{bmatrix}
                                       \bm{I}_a & \mathbf{0} \nonumber\\
                                       \pd{\bm{h}_{l-1,b}}{\bm{h}_{l,a}} & \mathrm{diag}(\bm{m}_a(\bm{h}_{l,a}))
                                     \end{bmatrix}\nonumber\\
                                   &= \det \left( \mathrm{diag}(\bm{m}_a(\bm{h}_{l,a})) \right),
\end{align}
where $\bm{I}_a$ is identity matrix and $\mathrm{diag}(\cdot)$ returns a square matrix with the elements of $(\cdot)$ on the main diagnal.
\textcolor{red}{do I still need this?}
Then the pdf is
\begin{align}
  p(\bm{x}) & =  p(\bm{z}) \big| \mathrm{det}(\bm{J}) |_{\bm{z}={\bm{f}}({\bm{x}})}\big| \nonumber\\
            &  = p(\bm{z}) \prod_{l=1}^L \abs{\det\left( \mathrm{diag}(\bm{m}_a(\bm{h}_{l,a}))]  \right)}.
\end{align}
[] describes a \textit{coupling} mapping between layers. Since the coupling has a partial identity mapping, direct concatenation of multiple such coupling mappings would result in a partial identity mapping of the whole neural network $\tilde{\bm{g}}$. Alternating the positions of identity mapping \cite{2016arXiv160508803D} or using $1\times1$ convolution operations \cite{2018arXiv180703039K} before each coupling mapping is used to treat the issue.
% \textcolor{blue}{Some reading problem in this sentence: Since the coupling has a partial identity mapping, alternating the positions of identity mapping \cite{2016arXiv160508803D} or using $1\times1$ convolution operations \cite{2018arXiv180703039K} when
% concatenating multiple layers together were proposed to avoid the situation that a part of the output is equal to a part of the input.} %In addition, applying affine mapping does not affect its advantage of trivial inverse and determinate computation.
Furthermore, \cite{2016arXiv160508803D}\cite{2018arXiv180703039K} split some hidden
layer signal $\bm{h}$ and model a part of it directly as standard Gaussian to reduce computation and memory burden.
% % \textcolor{blue}{Therefore the dimension
% of latent signal $\bm{z}$ to model can be smaller than $N$.}

\subsection{Learning of GenHMM}\label{subsec:optmGenHMM}

\subsubsection{Generative Model Update}
We use notation $\bm{\Pi} = \left\{  \bm{\pi}_{s}| s\in \Ss \right\}$, $\bm{\Theta}=\left\{ \bm{\theta}_s| s\in \Ss \right\}$.
Then the third subproblem, \eqref{eq:m-step-subs}, becomes:
\begin{align}\label{eq:sub-gm}
  &\umax{\bm{\Phi}} \Qq(\bm{\Phi}; \bm{H}^{\mathrm{old}}) = \umax{\bm{\Pi}} \Qq(\bm{\Pi}; \bm{H}^{\mathrm{old}}) + \umax{\bm{\Theta}} \Qq(\bm{\Theta}; \bm{H}^{\mathrm{old}}),
\end{align}
where
\begin{align}
  &\Qq(\bm{\Pi}; \bm{H}^{\mathrm{old}})  =\EE_{\hat{p}(\ubm{x}),p(\ubm{s},\ubm{\kappa}| \ubm{x}; \bm{H}^{\mathrm{old}})}\left[  \log\,p(\ubm{\kappa}| \ubm{s}; \bm{H})\right] \nonumber \\
  &\Qq(\bm{\Theta}; \bm{H}^{\mathrm{old}}) =\EE_{\hat{p}(\ubm{x}),p(\ubm{s},\ubm{\kappa}| \ubm{x}; \bm{H}^{\mathrm{old}})}\left[  \log\,p(\ubm{x}| \ubm{s},\ubm{\kappa}; \bm{H})\right] 
\end{align}

The neural network loss can be thus formulated as
\begin{align}\label{eq:obj-q-gen-mix}
  &\Qq(\bm{\Theta}; \bm{H}^{\mathrm{old}}) \nonumber \\
  = &\frac{1}{R}\sum_{r=1}^{R}\sum_{\ubmr{s}{r}}\sum_{\ubmr{\kappa}{r}}{p(\ubmr{s}{r}| \ubmr{x}{r}; \bm{H}^{\mathrm{old}})} \sum_{t=1}^{{T}^{r}}\log\,p(\bm{x}^{(r)} | \smtr{s}{t}{r}, \smtr{\kappa}{t}{r}; \bm{H}) \nonumber \\
  =& \frac{1}{R}\sum_{r=1}^{R} \sum_{t=1}^{{T}^{r}-1} \sum_{\smtr{s}{t}{r}=1}^{|\Ss|}  \sum_{\smtr{\kappa}{t}{r}=1}^{K}p(\smtr{s}{t}{r}| \ubmr{x}{r}; \bm{H}^{\mathrm{old}})p(\smtr{\kappa}{t}{r}|\smtr{s}{t}{r}, \ubmr{x}{r}; \bm{H}^{\mathrm{old}}) \nonumber\\
  &  \log\, p(\bmtr{x}{t}{r} | \smtr{s}{t}{r}, \smtr{\kappa}{t}{r}; \bm{H}) 
\end{align}


Here, $p(s_t| \ubm{x}, \bm{H}^{\mathrm{old}})$ is computed by forward/backward algorithm. The posterior of $\kappa$ is:
\begin{align}\label{eq:kappa-posterior}
  p(\kappa| s, \ubm{x}; \bm{H}^{\mathrm{old}})
  &=  \frac{p(\kappa, \ubm{x}| s; \bm{H}^{\mathrm{old}})}{p(\ubm{x}| s,\bm{H}^{\mathrm{old}})} \nonumber \\
  & = \frac{\pi_{s, \kappa}^{\mathrm{old}} p(\bm{x}| s, \kappa, \bm{H}^{\mathrm{old}})}{\sum_{\kappa=1}^{K}  \pi_{s, \kappa}^{\mathrm{old}} p(\bm{x}| s, \kappa,\bm{H}^{\mathrm{old}})} 
\end{align}
where the last equation is due to the fact that only $\bm{x}_t$ among sequence $\ubm{x}$ depends on $s_t, \kappa_t$. 

By substituting \eqref{eq:changel-variable} and \eqref{eq:cat-jacobian} into \eqref{eq:obj-q-gen-mix}, we have loss function for neural networks as
\begin{align}\label{eq:obj-q-gen-mix}
  &\Qq(\bm{\Theta}; \bm{H}^{\mathrm{old}}) \nonumber \\
  =& \frac{1}{R}\hspace{-3pt}\sum_{r=1}^{R}\hspace{-3pt} \sum_{t=1}^{{T}^{r}-1}\hspace{-3pt} \sum_{\smtr{s}{t}{r}=1}^{|\Ss|} \hspace{-3pt} \sum_{\smtr{\kappa}{t}{r}=1}^{K}p(\smtr{s}{t}{r}| \ubmr{x}{r}; \bm{H}^{\mathrm{old}})p(\smtr{\kappa}{t}{r}|\smtr{s}{t}{r}, \ubmr{x}{r}; \bm{H}^{\mathrm{old}}) \nonumber\\
  &\left[ \log\, p_{\smtr{s}{t}{r}, \smtr{\kappa}{t}{r}}(\bm{f}_{\smtr{s}{t}{r}, \smtr{\kappa}{t}{r}}(\bmtr{x}{t}{r})) + \sum_{l=1}^{L}\log\,| \det (\nabla{\bm{f}_{s,\kappa}^{[l]}})|\right]
\end{align}


In implementation, the generators of GenHMM uses standard Gaussian distribution for variables $\bm{z} = \bm{f}_{s,\kappa}(\bm{x})$, i.e. $p_{s,\kappa}(\bm{z})$ used standard Gaussian density function.

------------------------------


The latent prior for mixture of each source $s$ is obtained by solving the following problem:
\begin{align}\label{opm:pi}
  \pi_{s, \kappa} & = \uargmax{\pi_{s, \kappa}} \Qq(\bm{\Phi}; \bm{H}^{\mathrm{old}}) \\ \nonumber
                  & s.t. \, \sum_{\kappa=1}^{K} \pi_{s, \kappa}= 1, \forall s = 1, 2, \cdots, |\Ss|. 
\end{align}

To solve problem \eqref{opm:pi}, we formulate its Lagrange function as
\begin{equation}
  \Ll = \Qq(\bm{\Phi}; \bm{H}^{\mathrm{old}}) + \sum_{\lambda_s} \lambda_s\left( 1-  \sum_{\kappa=1}^{K}\pi_{s, \kappa}  \right).
\end{equation}
Solving $\pd{\Ll}{\pi_{s, \kappa}} = 0$ gives
\begin{equation}
  \pi_{s,\kappa} = \frac{1}{\lambda_s}\sum_{r=1}^{R} \sum_{t=1}^{{T}^{(r)}} p(\smtr{s}{t}{r}=s, \smtr{\kappa}{t}{r} =\kappa| \ubmr{x}{r}; \bm{H}^{\mathrm{old}})
\end{equation}
With condition $\sum_{\kappa=1}^{K} \pi_{s, \kappa}= 1, \forall s = 1, 2, \cdots, |\Ss|$, we have
\begin{equation}
  \lambda_s = \sum_{\kappa=1}^{K}\sum_{r=1}^{R} \sum_{t=1}^{{T}^{(r)}} p(\smtr{s}{t}{r}=s, \smtr{\kappa}{t}{r} =\kappa | \ubmr{x}{r}; \bm{H}^{\mathrm{old}})
\end{equation}
Then the solution to \eqref{opm:pi} is
\begin{equation}\label{eq:mix-latent-parameter-solution}
  \pi_{s, \kappa} = \frac{\sum_{r=1}^{R} \sum_{t=1}^{{T}^{r}} p(\smtr{s}{t}{r} =s, \smtr{\kappa}{t}{r}=\kappa | \ubmr{x}{r}; \bm{H}^{\mathrm{old}}) }{\sum_{k =1}^{K}\sum_{r=1}^{R} \sum_{t=1}^{{T}^{r}-1} p(\smtr{s}{t}{r} =s, \smtr{\kappa}{t}{r}=k | \ubmr{x}{r}; \bm{H}^{\mathrm{old}}) }
\end{equation}
where $p(s, \kappa | \ubm{x}; \bm{H}^{\mathrm{old}}) = p(s| \ubm{x}; \bm{H}^{\mathrm{old}}) p(\kappa | s, \ubm{x}; \bm{H}^{\mathrm{old}})$. $p(s| \ubm{x}; \bm{H}^{\mathrm{old}})$ that can be computed by results of forward/backward in Viterbi algorithm, while $p(\kappa | s, \ubm{x}; \bm{H}^{\mathrm{old}})$. $p(s| \ubm{x}; \bm{H}^{\mathrm{old}})$ is given by \eqref{eq:kappa-posterior}.

-----------------------

\subsubsection{Initial Probability Update}
\eqref{eq:init-distribution-update} can be written as:
\begin{align}
  &\Qq(\bm{q}; \bm{H}^{\mathrm{old}}) \nonumber\\
  &=\frac{1}{R} \sum_{r=1}^{R}\sum_{\ubmr{s}{r}} {p(\ubmr{s}{r}| \ubmr{x}{r}; \bm{H}^{\mathrm{old}})} \log\,p(\smtr{s}{1}{r}) \nonumber \\
  % & = \frac{1}{R}\sum_{r=1}^{R}\sum_{\smtr{s}{1}{r}=1}^{|\Ss|}\sum_{\smtr{s}{2}{r}=1}^{|\Ss|}\cdots \sum_{\smtr{s}{T^{r}}{r}}^{{|\Ss|}} {p(\smtr{s}{1}{r}, \smtr{s}{2}{r}, \cdots, \smtr{s}{T^{r}}{r}| \ubmr{x}{r}; \bm{H}^{\mathrm{old}})} \log\,p(\smtr{s}{1}{r}) \\
  & = \frac{1}{R}\sum_{r=1}^{R}\sum_{\smtr{s}{1}{r}=1}^{|\Ss|}{p(\smtr{s}{1}{r}| \ubmr{x}{r}; \bm{H}^{\mathrm{old}})} \log\,p(\smtr{s}{1}{r}) 
\end{align}

Since $p(\smtr{s}{1}{r})$ is the probability of initial state of GenHMM $\bm{H}$ for $r$-th sequential sample, actually $q_i = p({s}_{1} =i) $, $i= 1, 2, \cdots, |\Ss|$ for $\bm{H}$. Solution to problem:
\begin{align}
  \bm{q} &= \uargmax{\bm{q}} \Qq(\bm{q}; \bm{H}^{\mathrm{old}}), \nonumber \\
         &\mathrm{s.t.} \sum_{i=1}^{ |\Ss| }q_i = 1, q_i \geq 0, \forall s.
\end{align}
is
\begin{equation}\label{eq:update-initial-state-prob}
  q_i = \frac{1}{R} \sum_{r=1}^{R} p(\smtr{s}{1}{r}=i | \ubmr{x}{r}; \bm{H}^{\mathrm{old}}), \forall\; i = 1, 2, \cdots, |\Ss|.
\end{equation}

\subsubsection{Transition Probability Update}
\eqref{eq:transition-update} can be written as
\begin{align}
  &\Qq(\bm{A}; \bm{H}^{\mathrm{old}})\nonumber \\
  % &= \sum_{r=1}^{R} \EE_{p(\ubmr{s}{r}| \ubmr{x}{r}; \bm{H}^{\mathrm{old}})} \left[\log\,\sum_{t=1}^{T^{(r)}-1}p(\smtr{s}{t+1}{r}|\smtr{s}{t}{r}; {A})\right] \nonumber\\
  &= \sum_{r=1}^{R} \sum_{\ubmr{s}{r}}{p(\ubmr{s}{r}| \ubmr{x}{r}; \bm{H}^{\mathrm{old}})} \sum_{t=1}^{T^{(r)}-1}\log\,p(\smtr{s}{t+1}{r}|\smtr{s}{t}{r}; {A}) \nonumber \\
  &= \sum_{r=1}^{R} \sum_{t=1}^{T^{(r)}-1} \sum_{\smtr{s}{t}{r}=1}^{|\Ss|}\sum_{\smtr{s}{t+1}{r}=1}^{|\Ss|}{p(\smtr{s}{t}{r}, \smtr{s}{t+1}{r}| \ubmr{x}{r}; \bm{H}^{\mathrm{old}})} \log\,p(\smtr{s}{t+1}{r}|\smtr{s}{t}{r}; {A})
\end{align}

Since $\bm{A}_{i, j}  = p(\smtr{s}{t+1}{r}=j|\smtr{s}{t}{r}=i; {A})$ where $A_{i, j}$ is the element of transition matrix $A$, the solution to problem:
\begin{align}\label{eq:update-transition-prob}
  \bm{A} = &\uargmax{\bm{A}} \Qq(\bm{A}; \bm{H}^{\mathrm{old}}), \nonumber \\
  \mathrm{s.t.} &\hspace{0.2cm} \bm{A} \cdot \bm{1} = \bm{1} \nonumber \\
           &  \bm{A}^{\intercal} \cdot \bm{1} = \bm{1} \nonumber \\
           & \bm{A}_{i,j} \geq 0.
\end{align}
is
\begin{equation}
  \bm{A}_{i,j} = \frac{\bar{\xi}_{i,j}}{\sum_{k = 1}^{|\Ss|} \bar{\xi}_{i,k}},
\end{equation}
where
\begin{equation}\label{eq:update-transition-solt}
  \bar{\xi}_{i,j} = \sum_{r= 1}^{R} \sum_{t= 1}^{T^{(r)}-1}{p(\smtr{s}{t}{r}=i, \smtr{s}{t+1}{r}=j| \ubmr{x}{r}; \bm{H}^{\mathrm{old}})}
\end{equation}

\subsection{Algorithm of GenHMM}
We goal is to minimize the KL divergence $KL(\hat{p}(\ubm{x})\| p(\ubm{x};\bm{H}))$ or maximizing$\sum_{r=1}^{R}\log\,p(\ubmr{x}{r}; \bm{H})$. The 
Algorithm here:
things to talk:
\begin{itemize}
\item separate optimization of different sets of parameters
\item batch optimization optimization
\item not going to optimal for $\bm{\Phi}$, leaving to convergence analysis in next section
\end{itemize}


We summarize the optimization algorithm as:

\begin{algorithm}[H]
  \caption{Learning of GenHMM}
  \begin{algorithmic}[1]
    \STATE {\bfseries Input:}{
      Building $\bm{H}^{\mathrm{old}}, \bm{H} \in \Hh$ gives: \\
      $\bm{H}^{\mathrm{old}} = \{\Ss, \bm{q}^{\mathrm{old}}, A^{\mathrm{old}}, p(\bm{x}|s; \bm{\Phi}^{\mathrm{old}})\}$, $\bm{H} = \{\Ss, \bm{q}, A, p(\bm{x}|s; \bm{\Phi})\}$\\
    Empirical distribution $\hat{p}(\bm{x})$ of dataset;} \\
    \STATE Initialize $\bm{H}$
    \STATE $\bm{H}^{\mathrm{old}} \gets \bm{H}$
    \STATE Set learning rate $\eta$, neural network optimization epochs $T$ per EM step.
    \FOR { $\bm{H}$ not converge}
        \FOR {epoch $t < T$}
    \STATE Sample a batch of data $\left\{ \ubmr{x}{r} \right\}_{r=1}^{R_b}$ from dataset $\hat{p}(\ubm{x})$
    
    \STATE Compute posterior $p(\smtr{s}{t}{r}, \smtr{\kappa}{t}{r}| \ubmr{x}{r}; \bm{H}^{\mathrm{old}})$  
    \STATE Formulate loss ${\Qq}\left({\bm{\Theta}}, {\bm{H}}^{\mathrm{old}}\right)$ in \eqref{eq:obj-q-gen-mix}

    \STATE $\partial{\bm{\Theta}} \gets  \nabla_{\bm{\Theta}} {\Qq}\left({\bm{\Theta}},{\bm{H}}^{\mathrm{old}}\right)$
    \STATE $\bm{\Theta} \gets \bm{\Theta} + \eta \cdot \partial{\bm{\theta}_s}$
    \ENDFOR
    \STATE $\bm{q} \gets \uargmax{\bm{q}}\, \Qq(\bm{q}; \bm{H}^{\mathrm{old}})$ by \eqref{eq:update-initial-state-prob};
    \STATE $\bm{A} \gets \uargmax{\bm{A}}\Qq(\bm{A}; \bm{H}^{\mathrm{old}})$ by \eqref{eq:update-transition-prob};
    \STATE $\bm{\Pi} \gets \uargmax{\bm{\Pi}}\Qq(\bm{\Phi}; \bm{H}^{\mathrm{old}})$ by \eqref{eq:update-transition-solt};
    \STATE $\bm{H}^{\mathrm{old}} \gets \bm{H}$

    \ENDFOR
  \end{algorithmic}
\end{algorithm}






\subsection{Convergence Analysis}
\begin{prop}
  Assume that parameters of GenHMM are in a compact set. Further assume that $\Qq(\bm{\Theta}^{\mathrm{new}}; \bm{H}^{\mathrm{old}}) \geq \Qq(\bm{\Theta}^{\mathrm{old}}; \bm{H}^{\mathrm{old}})$, then we always have $KL(\hat{p}(\ubm{x})\| p(\ubm{x};\bm{H}^{\mathrm{new}})) \geq KL(\hat{p}(\ubm{x})\| p(\ubm{x};\bm{H}^{\mathrm{old}}))$, and GenHMM converges.
\end{prop}

\begin{proof}
  We begin with the comparison of log-likelihood evaluated under $\bm{H}^{\mathrm{old}}$ and $\bm{H}^{\mathrm{old}}$. The log-likelihood of dataset given by $\hat{p}(\ubm{x})$ can be reformulated as
  \begin{align*}
    &\EE_{\hat{p}(\ubm{x})}\left[ \log\,p(\ubm{x};\bm{H}^{\mathrm{new}}) \right] \nonumber \\
    =& \EE_{\hat{p}(\ubm{x}),p(\ubm{s},\ubm{\kappa}| \ubm{x}; \bm{H}^{\mathrm{old}})}\left[ \log\,\frac{p(\ubm{x}, \ubm{s}, \ubm{\kappa}; \bm{H}^{\mathrm{new}})}{p(\ubm{s}, \ubm{\kappa}|\ubm{x}; \bm{H}^{\mathrm{old}})}\right] \nonumber \\
    &+ \EE_{\hat{p}(\ubm{x})}\left[ KL(p(\ubm{s}, \ubm{\kappa}|\ubm{x}; \bm{H}^{\mathrm{old}})\|p(\ubm{s}, \ubm{\kappa}|\ubm{x}; \bm{H}^{\mathrm{new}})) \right],
  \end{align*}
  where the first term on the right hand side of equality can be further written as
  \begin{align*}
    &\EE_{\hat{p}(\ubm{x}),p(\ubm{s},\ubm{\kappa}| \ubm{x}; \bm{H}^{\mathrm{old}})}\left[ \log\,\frac{p(\ubm{x}, \ubm{s}, \ubm{\kappa}; \bm{H}^{\mathrm{new}})}{p(\ubm{s}, \ubm{\kappa}|\ubm{x}; \bm{H}^{\mathrm{old}})}\right] \nonumber \\
    = &\Qq(\bm{H}^{\mathrm{new}}; \bm{H}^{\mathrm{old}}) + \EE_{\hat{p}(\ubm{x}),p(\ubm{s},\ubm{\kappa}| \ubm{x}; \bm{H}^{\mathrm{old}})}\left[p(\ubm{s}, \ubm{\kappa}|\ubm{x}; \bm{H}^{\mathrm{old}})\right].
  \end{align*}
  According to subsection~\ref{subsec:optmGenHMM}, the subproblem solutions gives
  \begin{align*}
    \Qq(\bm{q}^{\mathrm{new}}; \bm{H}^{\mathrm{old}}) &\geq \Qq(\bm{q}^{\mathrm{old}}; \bm{H}^{\mathrm{old}}),\nonumber \\
    \Qq(\bm{A}^{\mathrm{new}}; \bm{H}^{\mathrm{old}}) &\geq \Qq(\bm{A}^{\mathrm{old}}; \bm{H}^{\mathrm{old}}),\nonumber \\
    \Qq(\bm{\Pi}^{\mathrm{new}}; \bm{H}^{\mathrm{old}}) &\geq \Qq(\bm{\Pi}^{\mathrm{old}}; \bm{H}^{\mathrm{old}}).
  \end{align*}
  For the learning w.r.t. neural network parameter set $\bm{\Theta}$, as long as the lose function does not decrease during EM iterations, i.e. $\Qq(\bm{\Theta}^{\mathrm{new}}; \bm{H}^{\mathrm{old}}) \geq \Qq(\bm{\Theta}^{\mathrm{old}}; \bm{H}^{\mathrm{old}})$, we would have
  \begin{equation*}
    \Qq(\bm{H}^{\mathrm{new}}; \bm{H}^{\mathrm{old}}) \geq \Qq(\bm{H}^{\mathrm{old}}; \bm{H}^{\mathrm{old}}),
  \end{equation*}
  since
  \begin{align*}
    \Qq(\bm{H}^{\mathrm{new}}; \bm{H}^{\mathrm{old}}) = &\Qq(\bm{q}^{\mathrm{new}}; \bm{H}^{\mathrm{old}}) + \Qq(\bm{A}^{\mathrm{new}}; \bm{H}^{\mathrm{old}}) \nonumber \\
    &+ \Qq(\bm{\Pi}^{\mathrm{new}}; \bm{H}^{\mathrm{old}})
    + \Qq(\bm{\Theta}^{\mathrm{new}}; \bm{H}^{\mathrm{old}}).
  \end{align*}
  Therefore we have the inequality 
  \begin{align*}
    &\EE_{\hat{p}(\ubm{x}),p(\ubm{s},\ubm{\kappa}| \ubm{x}; \bm{H}^{\mathrm{old}})}\left[ \log\,\frac{p(\ubm{x}, \ubm{s}, \ubm{\kappa}; \bm{H}^{\mathrm{new}})}{p(\ubm{s}, \ubm{\kappa}|\ubm{x}; \bm{H}^{\mathrm{old}})}\right] \nonumber \\
    \geq &\EE_{\hat{p}(\ubm{x}),p(\ubm{s},\ubm{\kappa}| \ubm{x}; \bm{H}^{\mathrm{old}})}\left[ \log\,\frac{p(\ubm{x}, \ubm{s}, \ubm{\kappa}; \bm{H}^{\mathrm{old}})}{p(\ubm{s}, \ubm{\kappa}|\ubm{x}; \bm{H}^{\mathrm{old}})}\right].
  \end{align*}
  Due to the fact that $KL(p(\ubm{s}, \ubm{\kappa}|\ubm{x}; \bm{H}^{\mathrm{old}})\|p(\ubm{s}, \ubm{\kappa}|\ubm{x}; \bm{H}^{\mathrm{old}}))=0$, we have
  \begin{align*}
      &\EE_{\hat{p}(\ubm{x})}\left[ \log\,p(\ubm{x};\bm{H}^{\mathrm{new}}) \right] \nonumber \\
    \geq & \EE_{\hat{p}(\ubm{x}),p(\ubm{s},\ubm{\kappa}| \ubm{x}; \bm{H}^{\mathrm{old}})}\left[ \log\,\frac{p(\ubm{x}, \ubm{s}, \ubm{\kappa}; \bm{H}^{\mathrm{old}})}{p(\ubm{s}, \ubm{\kappa}|\ubm{x}; \bm{H}^{\mathrm{old}})}\right] \nonumber \\
    = & \EE_{\hat{p}(\ubm{x})}\left[ \log\,p(\ubm{x};\bm{H}^{\mathrm{old}}) \right].
  \end{align*}
  Therefore
  \begin{align*}
    KL(\hat{p}(\ubm{x})\|p(\ubm{x};\bm{H}^{\mathrm{new}})) = &\EE_{\hat{p}(\ubm{x})}\left[ \log\,\frac{\hat{p}(\ubm{x})}{p(\ubm{x};\bm{H}^{\mathrm{new}})} \right] \nonumber \\
    \leq &\EE_{\hat{p}(\ubm{x})}\left[ \log\,\frac{\hat{p}(\ubm{x})}{p(\ubm{x};\bm{H}^{\mathrm{old}})} \right] \\
    = & KL(\hat{p}(\ubm{x})\|p(\ubm{x};\bm{H}^{\mathrm{old}})).
    \end{align*}
  Since KL divergence is non-negative and thus lower bounded, GenHMM will converge.
\end{proof}

\section{Experiments}
test citing \cite{2018arXiv180703039K}

For problem \eqref{eq:sub-gm} we are going to use our generative models to solve. I have the following consideration to revised our LatMM and GenMM for this application:
\begin{itemize}
\item Use factorized model instead of additive mixture model, to make likelihood computation logarithm domain compatible; 
\item Use full EM fashion instead of mini-batch fashion for training: store generative model as old for EM, there are always two neural networks working, one old for probability evaluation and one new for optimization.
\end{itemize}

\bibliography{myref}
\bibliographystyle{aaai}

% \bibliography{bibliography}


% \appendix

% \input{section/sec-appendixA}
\end{document}


%%% Local Variables:
%%% mode: latex
%%% TeX-master: t
%%% End:
